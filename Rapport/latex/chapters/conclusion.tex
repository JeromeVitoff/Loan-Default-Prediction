% !TEX root = manuscript.tex
\clearpage
\chapter*{Conclusion}
\vspace{-1cm}
Ce projet s'inscrit dans le cadre de la prédiction du défaut de paiement des prêts (Loan Default Prediction), un domaine d'une importance capitale pour le secteur financier. En permettant aux institutions bancaires et financières de prédire avec précision les risques de défaut, ce projet peut contribuer à une meilleure gestion des risques, à la réduction des pertes et à une prise de décision plus éclairée. L'approche par machine learning utilisée dans ce projet s'appuie sur des techniques avancées telles que l'analyse des données clients, leur historique de crédit et d'autres facteurs économiques et sociaux pour créer des modèles de prédiction robustes.

Cette étude a permis de comparer plusieurs modèles de machine learning dans le cadre d'un problème de classification, en évaluant leurs performances à travers des métriques adaptées telles que l'accuracy et le F1-score macro. Les résultats ont montré que les modèles basés sur les forêts aléatoires et la régression logistique sont particulièrement efficaces pour gérer les défis posés par les données déséquilibrées, chaque modèle ayant ses propres forces et limites.

L'approche de validation croisée a été essentielle pour garantir une évaluation rigoureuse des modèles et éviter les biais de sur-apprentissage, tout en assurant une meilleure généralisation des résultats sur de nouveaux ensembles de données. Malgré les performances encourageantes de certains modèles, il est apparu que des améliorations étaient possibles, notamment en ajustant les hyperparamètres ou en explorant des techniques plus sophistiquées pour gérer le déséquilibre des classes.

Cependant, certaines limitations demeurent dans cette étude, notamment la gestion du recall pour les classes minoritaires, un aspect qui mérite une attention particulière pour une optimisation complète des modèles. De plus, l'expérimentation d'autres algorithmes, tels que des méthodes d'ensemble plus avancées ou des réseaux de neurones, pourrait offrir des performances améliorées.

Pour des travaux futurs, il serait intéressant d'élargir cette analyse en intégrant davantage de techniques de pré-traitement des données, ainsi que l'optimisation des modèles à travers des approches comme la recherche bayésienne des hyperparamètres. Une meilleure gestion de l'équilibre entre les classes pourrait aussi constituer un axe d'amélioration important pour maximiser l'efficacité des prédictions sur les classes minoritaires.

En conclusion, cette étude met en lumière l'importance de choisir judicieusement les modèles et les métriques, en fonction des spécificités des données et des objectifs de la tâche. Les résultats obtenus ouvrent la voie à de nouvelles explorations et à l'amélioration continue des modèles pour des applications réelles.

